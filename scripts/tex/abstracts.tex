\hypertarget{DashaGloutak}{\subsection*{\color{CUGOLD} Unsteady Loading Of A Wing In Global Streamwise Gusts}} \vsp 
\underline{Dasha Gloutak}, \textit{Aerospace Engineering, University Of Colorado Boulder}\\ 
{Kenneth Jansen}, \textit{Aerospace Engineering, University Of Colorado Boulder}\\ 
{John Farnsworth}, \textit{Aerospace Engineering, University Of Colorado Boulder}\\ 
\vspace{-0.1 in} \\ 
\noindent Characterizing the aerodynamic response of wings to oncoming gusts is critical to maintaining stability and efficiency of aircraft. In this study, surface pressure and particle image velocimetry measurements are used to analyze the unsteady flow physics of a separated NACA 0015 wing in global streamwise gusts, which impose a time-varying velocity on the wing. Unsteadiness exhibited in the wing’s aerodynamic response to velocity acceleration and deceleration can be attributed to the dynamics of developing vortical structures on the suction side. Whether the flow is accelerating or decelerating determines the temporal and spatial scales of the vortical structures, including the convective time, size, and location from which vortical structures develop and shed. These scales determine the degree to which vortical structures interact with each other and with the wing surface, thereby also influencing the unsteady loading on the wing. \\ 
\noindent  \\ 
\noindent  *This material is based upon work supported by the Air Force Office of Scientific Research under Award Numbers FA9550-18-1-0311 and FA9550-21-1-0133. \\ 
\begin{flushright}\vspace{-0.2 in}\hyperlink{toc}{Back to table of contents}\end{flushright}\vspace{-0.2 in}
\hypertarget{LarsLarson}{\subsection*{\color{CUGOLD} Experimental And Numerical Characterization Of Odor Plume Structure In The Wake Of A Commercial Odor-Delivery Device}} \vsp 
\underline{Lars Larson}, \textit{CEAE, University Of Colorado Boulder}\\ 
{Anna Pauls}, \textit{CEAE, University Of Colorado Boulder}\\ 
{John Crimaldi}, \textit{CEAE, University Of Colorado Boulder}\\ 
\vspace{-0.1 in} \\ 
\noindent We conduct experiments and simulations to quantify the spatiotemporal dynamics of odor plumes forming in the wake of a Training Aid Delivery Device (TADD). The TADD is a cylindrical glass jar topped with an odor-permeable membrane created for the US Army as an aid for training military working dogs for scent localization. For the experiment, we use a photo-ionization detector (PID) instrument that is sensitive to vapors from volatile organic compounds (VOCs) to map odor fields emanating from a TADD containing liquid acetone located on the floor of a low-speed wind tunnel. We investigate the effect of mean crossflow velocity in the tunnel and compare the results with numerical simulations. The simulations consist of a coupled solution of incompressible Navier-Stokes and advection-diffusion equations. We emulate an acetone vapor plume to calibrate the model to the experimental results, then simulate odor plumes for a range of crossflow velocities and odorant specific gravities (SG). For SG=1, the odor plume is unimodal with peak concentrations on the centerline directly downstream. Heavy plumes (SG>1) interact with the horseshoe vortex that forms around the TADD, creating a structured odor plume with bimodal peaks away from the centerline and lower concentrations directly downwind. \\ 
\begin{flushright}\vspace{-0.2 in}\hyperlink{toc}{Back to table of contents}\end{flushright}\vspace{-0.2 in}
\hypertarget{LukasSpies}{\subsection*{\color{CUGOLD} Comparison Of Some Rans Solvers}} \vsp 
\underline{Lukas Spies}, \textit{National Renewable Energy Laboratory}\\ 
{Ethan Young}, \textit{National Renewable Energy Laboratory}\\ 
{Jeffery Allen}, \textit{National Renewable Energy Laboratory}\\ 
\vspace{-0.1 in} \\ 
\noindent We will take a look at solving the Reynolds-averaged Navier-Stokes (RANS) equations that are encountered in the context of wind farm performance simulations and optimizations. We will compare some of the more popular ways to solve these equations with a focus on using iterative solvers for the linear solve. We will compare their performance and reliability to a direct solve as we scale the problem both by adding more and parallel resources and by increasing the size of the domain, both in two and three dimensions. \\ 
\noindent  \\ 
\noindent There are many strategies that can be applied to solving the RANS equations, some are very efficient, while others are very insensitive to, for example, the Reynolds number. The first contender we will consider is the Pressure-Convection-Diffusion (PCD) preconditioner. Early results suggest that PCD is indeed a very efficient solver, in particular in two dimensions, as long as the Reynold's number remains small. Next we will try to reorder our degrees of freedom such that we can use GMRES with ILU for our linear solve. Another popular choice we will consider for solving the RANS equations is SIMPLE (and its derivatives). \\ 
\noindent  \\ 
\noindent For all of our implementations we make use of either FEniCS or Firedrake, basing our work on both existing and new implementations. \\ 
\begin{flushright}\vspace{-0.2 in}\hyperlink{toc}{Back to table of contents}\end{flushright}\vspace{-0.2 in}
\hypertarget{AaronTrue}{\subsection*{\color{CUGOLD} Distortion Of Passive Scalar Structure During Suction-Based Plume Sampling}} \vsp 
\underline{Aaron True}, \textit{CEAE, University Of Colorado Boulder}\\ 
{John Crimaldi}, \textit{CEAE, University Of Colorado Boulder}\\ 
\vspace{-0.1 in} \\ 
\noindent Studies of plume dynamics often rely on photoionization detectors (PID) to quantify spatiotemporal distributions of passive scalars (gases, vapors, odors). However, the potential for PID suction to distort filaments and to modify sensed time records remains unclear. We used computational fluid dynamics to model a common PID to quantify and parameterize suction distortion by considering how sensed time records compare to those registered by an ideal probe. Models cover a range of realistic plumes, and we show that PID can significantly modify the peak concentration and pulse shape of sensed records. We quantified distortion variations in three nondimensional parameters describing PID geometry and sampling conditions: relative suction rate, relative filament size, and ambient flow Reynolds number. We used analytical models, dimensional analysis, and scaling arguments to interpret results and discuss when distortion is likely and what drives it. We built dimensionless distortion prediction regressions, and our results enable PID users to estimate distortion levels and to employ mitigation strategies through suction velocity tuning. These findings can inform distortion-mitigating design principles and best sampling practices for other suction-based passive scalar sensing schemes. \\ 
\begin{flushright}\vspace{-0.2 in}\hyperlink{toc}{Back to table of contents}\end{flushright}\vspace{-0.2 in}
\hypertarget{JaylonMcghee}{\subsection*{\color{CUGOLD} Impact Of Canonical Perturbations In The Inflow On Wind Turbine Loads}} \vsp 
\underline{Jaylon Mcghee}, \textit{Aerospace Engineering, University Of Colorado Boulder}\\ 
{Divya Wagh}, \textit{Mechanical Engineering, Rice University}\\ 
{Ganesh Vijayakumar}, \textit{National Renewable Energy Laboratory}\\ 
{John Farnsworth}, \textit{Aerospace Engineering, University Of Colorado Boulder}\\ 
\vspace{-0.1 in} \\ 
\noindent Characterizing the complex relationship between wind turbines and Earth’s atmospheric boundary layer remains a challenge in wind energy research. In particular, the impact of the wide range of spatial and temporal velocity scales present within the atmospheric boundary layer and the subsequent impact they have on dynamic loads of wind turbine loads is not fully understood. This study analyzes the response of wind turbines to canonical inflows with a spatial or temporal oscillation in inflow velocity field. Individual blade sections are demonstrated to experience highly unsteady angle of attack variations from spatial waveforms in the inflow, potentially leading to undesired load effects. Temporally varying inflow fluctuations produce integrated loads that are nearly an order of magnitude higher than the response to spatial oscillations. Further investigation of the load response to canonical temporal or spatial oscillations can facilitate more resilient turbine design by reducing the number of critical simulation cases and helping wind turbine designers identify flows of interest.  \\ 
\begin{flushright}\vspace{-0.2 in}\hyperlink{toc}{Back to table of contents}\end{flushright}\vspace{-0.2 in}
\hypertarget{RobertSasse}{\subsection*{\color{CUGOLD} Development And Application For Uas}} \vsp 
\underline{Robert Sasse}, \textit{Aerospace Engineering, University Of Colorado Boulder}\\ 
\vspace{-0.1 in} \\ 
\noindent Uncrewed aircraft systems (UAS) are deployed to collect high quality in-situ data for key atmospheric variables. The small size and maneuverability of UAS are unique advantages that allow for in-situ observation in otherwise inaccessible locations.  \\ 
\noindent  \\ 
\noindent Researchers use UAS technology for various projects including wind turbine studies. UAS are ideal to study turbine wakes created around wind farms and can help improve the understanding of inter-turbine interactions. Another exciting potential application of UAS technology relates to studying shear induced turbulence in the upper oceans. Wind is one of the main forcing terms for ocean mixing and observation through UAS measurement is an important part of validating and informing ocean numerical models.     \\ 
\noindent  \\ 
\noindent Using UAS to measure wind can be achieved by several methods. One tool commonly used is the multi hole probe (MHP). These probes rely on differential pressure measurements as well as inertial measurements and GPS data to compute wind vectors. External sensors like MHPs can become occluded by water when it rains or be damaged during landings. An alternative could be an accelerometer matrix which would measure forces on a UAS airframe. Using thrust, orientation and heading information it is possible to determine wind components. \\ 
\noindent  \\ 
\begin{flushright}\vspace{-0.2 in}\hyperlink{toc}{Back to table of contents}\end{flushright}\vspace{-0.2 in}
\hypertarget{ThomasPuhr}{\subsection*{\color{CUGOLD} Designing A Benchtop Flow Loop For Investigating Particle Transport In Human Arterial Flows}} \vsp 
\underline{Thomas Puhr}, \textit{Mechanical Engineering, University Of Colorado Boulder}\\ 
{Argudit Chauhan}, \textit{Biomedical Engineering, University Of Colorado Boulder}\\ 
{Parker Mcdonnell}, \textit{Mechanical Engineering, University Of Colorado Boulder}\\ 
{Kaushik Jayaram}, \textit{Mechanical Engineering, University Of Colorado Boulder}\\ 
{Nick Bottenus}, \textit{Mechanical Engineering, University Of Colorado Boulder}\\ 
{Debanjan Mukherjee}, \textit{Mechanical Engineering, University Of Colorado Boulder}\\ 
\vspace{-0.1 in} \\ 
\noindent Particle transport and distribution in pulsatile arterial flow environments has major implications in physiological phenomena in health and disease, including stroke and thrombosis, and drug delivery. As tracking particles in vivo remains intractable, and in silico models have associated assumptions and limitations, development of physiological benchtop flow-loop models remains of interest. Here, we discuss the design of an in vitro benchtop flow loop capable of tracking flow and transport of particles across 3D printed models of human arterial segments. We outline the various components of the loop, and focus on design specifics regarding inflow control, outflow control, particle selection, particle release and visualization. To illustrate the utility and efficacy of our in vitro benchtop design, we illustrate a series of embolic particle transport studies across an idealized model equivalent of the human common carotid artery bifurcation. Specifically, we quantify distribution of embolic particles across the bifurcation model, for varying particle sizes, densities, and flow rates. Specific design features, and design optimizations, for the benchtop flow loop will also be discussed as part of ongoing research efforts.   \\ 
\begin{flushright}\vspace{-0.2 in}\hyperlink{toc}{Back to table of contents}\end{flushright}\vspace{-0.2 in}
\hypertarget{SebastianLaudenschlager}{\subsection*{\color{CUGOLD} Estimation Of Pulmonary Vascular Impedance For Children With Single Ventricle}} \vsp 
\underline{Sebastian Laudenschlager}, \textit{Radiology, University Of Colorado Anschutz}\\ 
{Mehdi Hedjazi Moghari}, \textit{Radiology, University Of Colorado Anschutz}\\ 
{Vitaly Kheyfets}, \textit{Pediatrics, University Of Colorado Anschutz}\\ 
\vspace{-0.1 in} \\ 
\noindent Children with single ventricle heart disease require three surgical procedures to separate oxygenated from deoxygenated blood flow and improve blood flow circulation. In the last procedure, Fontan, inferior vena cava is connected to pulmonary arteries to allow deoxygenated blood to passively flow to the pulmonary circuit. To determine an optimal anastomosis site and diameter, left and right pulmonary vascular impedance (PVI) needs to be estimated. Thus, the aim of this project is to estimate PVI in both lungs for single ventricle patients who are referred for Fontan operations. Magnetic resonance imaging was used to generate a patient-specific 3D model of the superior vena cava (SVC) and the left and right pulmonary arteries in ten patients. The 3D model was used in a novel computational fluid dynamic (CFD) optimization pipeline to estimate PVIs, where PVIs are modeled by three-element Windkessel parameters. Measured blood flow at the SVC was used as inlet flow, while measured blood flow at the left and right pulmonary arteries (outlets) was used in the optimization routine, along with mean SVC pressure measured by catheterization. A simplified 0D model was then developed, and the estimated PVIs using CFD and the 0D model are shown to be comparable. \\ 
\begin{flushright}\vspace{-0.2 in}\hyperlink{toc}{Back to table of contents}\end{flushright}\vspace{-0.2 in}
\hypertarget{KellyCao}{\subsection*{\color{CUGOLD} Computational Hemodynamics Using 3D Rotational Angiography Imaging}} \vsp 
\underline{Kelly Cao}, \textit{Biomedical Engineering, University Of Colorado Boulder}\\ 
{Jenny Zablah}, \textit{Pediatrics, Children’S Hospital Of Colorado Heart Institute, University Of Colorado Anschutz Medical Campus}\\ 
{Michael Shorofsky}, \textit{Pediatrics, Children’S Hospital Of Colorado Heart Institute, University Of Colorado Anschutz Medical Campus}\\ 
{Debanjan Mukherjee}, \textit{Mechanical Engineering, University Of Colorado Boulder}\\ 
\vspace{-0.1 in} \\ 
\noindent Congenital heart disease treatment commonly uses 3D Rotational Angiography (3DRA) imaging. 3DRA imaging has advantages of having a higher resolution, no additional increase in radiation, and availability of catheter hemodynamic information. 3DRA imaging paired with CFD can provide physicians with a visualization of how blood flows through the region of interest involved in pediatric congenital heart surgeries pre- and post- surgery. These visualizations can provide information about ways of improving surgeries and minimizing impacts on patient health. However, 3DRA is not an imaging choice that has commonly been used for CFD analysis. In this study, we developed a computational blood flow modeling approach based on 3DRA imaging for pediatric congenital heart surgery. We conduct segmentation of vessels of interest from multi-planar reconstructed 3DRA images of two passive flow case studies – a Glenn procedure, and a Fontan procedure case. Each of these cases have different anatomies and physiologies to illustrate how models of varying complexity can be used for CFD. Simulation results for both cases will be presented, illustrating flow patterns, vorticity and helicity, and wall shear distribution.  \\ 
\begin{flushright}\vspace{-0.2 in}\hyperlink{toc}{Back to table of contents}\end{flushright}\vspace{-0.2 in}
\hypertarget{SummerAndrews}{\subsection*{\color{CUGOLD} Image Based In Silico Modeling Of Transarterial Radioembolization For Liver Cancer}} \vsp 
\underline{Summer Andrews}, \textit{Mechanical Engineering, University Of Colorado Boulder}\\ 
{Premal Trivedi}, \textit{Interventional Radiology, University Of Colorado Anschutz}\\ 
{Debanjan Mukherjee}, \textit{Mechanical Engineering, University Of Colorado Boulder}\\ 
\vspace{-0.1 in} \\ 
\noindent Transarterial radioembolization (TARE) is a procedure in which radioactive Y90 particles are delivered intra-arterially to a tumor to treat liver cancer. Treatment accuracy is dependent on blood flow conditions in the hepatic arteries and the position of the catheter used for delivery. Clinicians conduct pretreatment mapping studies in which a surrogate particle is injected into the patient to determine optimal catheter position. However, this method does not fully characterize Y90 particle behavior during actual treatment. The aim of this study was to utilize standard-of-care imaging modalities to develop an in silico patient-specific model to simulate pathological conditions during TARE. Flow ratio boundary conditions were derived from patient Computed Tomography Angiography (CTA), Cone Beam Computed Tomography (CT), and Digital Subtraction Angiography (DSA) images. Flow in arteries was computed using SimVascular. Particles were injected via a modeled catheter and a previously developed Lagrangian tool kit. Simulated flow into tumor-feeding vessels was reflective of preferential blood flow observed in vivo. This workflow can be used to study how catheter placement and injection, as well as particle morphology can alter drug distribution.  \\ 
\begin{flushright}\vspace{-0.2 in}\hyperlink{toc}{Back to table of contents}\end{flushright}\vspace{-0.2 in}
\hypertarget{ChayutTeeraratkul}{\subsection*{\color{CUGOLD} Flow And Flow Mediated Transport In Dynamic Blood Clot Neighborhoods}} \vsp 
\underline{Chayut Teeraratkul}, \textit{Mechanical Engineering, University Of Colorado Boulder}\\ 
{Maurizio Tomaiuolo}, \textit{Wills Eye Hospital}\\ 
{Timothy J. Stalker}, \textit{Medicine, Thomas Jefferson University}\\ 
{Debanjan Mukherjee}, \textit{Mechanical Engineering, University Of Colorado Boulder}\\ 
\vspace{-0.1 in} \\ 
\noindent Blood flow and flow-mediated transport of biochemical species in the neighborhood of a blood clot are intimately connected to diseases such as stroke and heart attack. Comprehensive understanding of clot-flow interactions, and resulting transport phenomena, is precluded by several challenges. One key challenge lies in characterizing the dynamic flow environment around a realistic clot or arbitrary shape, porous microstructure, and changing shape over time. Here, we present a computational model that addresses this challenge, by directly integrating custom finite element multiphysics modeling with dynamic intravital microscopy imaging obtained from mouse injury models. The growing clot geometry manifold was identified and distinguished into a core and a shell region with differing microstructural properties, based on image processing and segmentation. This was coupled with a stabilized finite element flow and transport solver; and resulting flow and pressure data was used to illustrate the role of advection, diffusion, and permeation processes in a realistic dynamic clot neighborhood. \\ 
\begin{flushright}\vspace{-0.2 in}\hyperlink{toc}{Back to table of contents}\end{flushright}\vspace{-0.2 in}
\hypertarget{SreeparnaMajee}{\subsection*{\color{CUGOLD} Distance Field Based Approach For Resolving Particle-Wall Interactions For Biomedical Flows}} \vsp 
\underline{Sreeparna Majee}, \textit{Mechanical Engineering, University Of Colorado Boulder}\\ 
{Akshita Sahni}, \textit{University Of Colorado Anschutz}\\ 
{Ricardo Roopnarinesingh}, \textit{Mechanical Engineering, University Of Colorado Boulder}\\ 
{Aditya Balu}, \textit{Iowa State University}\\ 
{Adarsh Krishnamurthy}, \textit{Mechanical Engineering, Iowa State University}\\ 
{Debanjan Mukherjee}, \textit{Mechanical Engineering, University Of Colorado Boulder}\\ 
\vspace{-0.1 in} \\ 
\noindent In biomedical flows, particle transport analysis is a commonly used tool for studying flow phenomena with applications in diseases like stroke, thrombosis, drug delivery etc. However, modeling particle-wall interactions can often be difficult due to geometric and topological complexities of the biomedical flow domain. Here, we utilize a signed distance field (SDF) based approach to compute particle-wall collisions in Eulerian flow systems using a Lagrangian tracking method. SDF represents the distance of any point within the domain from its nearest boundary. Thus, collision is indicated when the distance value is less than the particle radius, and subsequently the contact event and resulting change in momentum is resolved. Our approach reduces the number of contact checks thereby reducing the complexity of the collision algorithm. We demonstrate two case-studies: (a) an idealized test verifying the effectiveness of our approach in terms of computational time, and (b) a real-world application for calculating particle trajectories in a flow field for patient-specific arterial geometries.  \\ 
\begin{flushright}\vspace{-0.2 in}\hyperlink{toc}{Back to table of contents}\end{flushright}\vspace{-0.2 in}
\hypertarget{ParneethLokini}{\subsection*{\color{CUGOLD} Laser Ignition And Laser-Induced Breakdown Spectroscopy Of A Hydrocarbon Flame In An Annular Spray Burner}} \vsp 
\underline{Parneeth Lokini}, \textit{Mechanical Engineering, Colorado State University}\\ 
{Bret Windom}, \textit{Mechanical Engineering, Colorado State University}\\ 
{Azer Yalin}, \textit{Mechanical Engineering, Colorado State University}\\ 
\vspace{-0.1 in} \\ 
\noindent Laser-induced ignition of n-heptane and air mixtures in an annular co-flow spray burner is investigated in this paper. Laser-induced breakdown spectroscopy (LIBS) is conducted simultaneously with the laser ignition to determine the local equivalence ratio by using past correlations based on the intensity ratio of the H and O atomic emission lines. The ignition location and global equivalence ratio are varied to understand the impact of local equivalence ratio and spray characteristics (e.g., mean droplet diameter and mean droplet velocity) on ignitability. The data acquired from this study can contribute to development of novel ignition techniques and validation of computational models for ignition and combustion in multiphase reacting flows. \\ 
\begin{flushright}\vspace{-0.2 in}\hyperlink{toc}{Back to table of contents}\end{flushright}\vspace{-0.2 in}
\hypertarget{IrisKessler}{\subsection*{\color{CUGOLD} Development Of Advanced Hydrogen Fueled Gas Turbine Combustion Systems}} \vsp 
\underline{Iris Kessler}, \textit{Mechanical Engineering, Colorado State University}\\ 
{Bret Windom}, \textit{Mechanical Engineering, Colorado State University}\\ 
{Miguel Valles}, \textit{Mechanical Engineering, Colorado State University}\\ 
\vspace{-0.1 in} \\ 
\noindent Gas turbines are used to generate large-scale thrust and work, and operate at efficiencies up to 15% higher than traditional IC engines. These devices are often fueled with fossil fuels and emit large quantities of GHG, and NOx. To meet environmental regulations, Colorado State is assisting Solar Turbines to create a low-emission industrial gas turbine system capable of operating high levels of hydrogen in Natural Gas mixtures. Hydrogen is an efficient source of thermochemical energy and can be produced through electrolysis systems powered by excess renewable electricity. When combusted, hydrogen produces H2O in place of CO2, making it a more favorable approach to limiting GHG. Concerns with this alternative fuel lie in hydrogens' high flame speed and temperature, which increases the risk of flashback and NOx production. To address this, a high fidelity and tractable chemical mechanism is needed to correctly model flame behavior, NOx emissions, autoignition, and extinction accurately. This will support the computational-aided design of next-generation combustion systems. This project will evaluate previously developed mechanisms on their accuracy in computing experimental flame speeds and ignition delay, as well as produce a reduced mechanism for use in advanced CFD simulations.  \\ 
\begin{flushright}\vspace{-0.2 in}\hyperlink{toc}{Back to table of contents}\end{flushright}\vspace{-0.2 in}
\hypertarget{MichaelMeehan}{\subsection*{\color{CUGOLD} The Role Of Diffusion And Viscosity On Laminar Unsteady Plumes}} \vsp 
\underline{Michael Meehan}, \textit{Mechanical Engineering, University Of Colorado Boulder}\\ 
{Peter Hamlington}, \textit{Mechanical Engineering, University Of Colorado Boulder}\\ 
\vspace{-0.1 in} \\ 
\noindent Buoyant plumes often exhibit a global instability that results in large-scale vortical structures at the base which can have subsequent implications on the downstream flow development. Numerical simulations have been an invaluable tool to study these structures, enabling exploration of parameter spaces that are difficult to access experimentally and examination of the underlying flow physics. In our previous work, direct numerical simulations of three-dimensional axisymmetric helium plume were conducted to understand the effect of the inlet-based Reynolds (Re) and Richardson (Ri) numbers on the flowfield. From these simulations, some fundamental questions still remain unanswered with respect to the laminar unsteady plumes in the limit of large Re, including: (i) do all laminar plumes transition to turbulence in the near-field, (ii) is the Strouhal number scaling most applicable at large Re, and (iii) do pressure effects on the downstream transport of fluid become negligible at large Re? In this presentation, we answer these questions by conducting a new simulation of identical buoyancy-related properties (e.g., density ratio, etc.) to the previous laminar unsteady helium plume but artificially set the viscosity and diffusion coefficients to zero. \\ 
\begin{flushright}\vspace{-0.2 in}\hyperlink{toc}{Back to table of contents}\end{flushright}\vspace{-0.2 in}
\hypertarget{AdamBinswanger}{\subsection*{\color{CUGOLD} Validation Of Exascale Combustion Code For The Simulation Of An Internal Combustion Engine}} \vsp 
\underline{Adam Binswanger}, \textit{National Renewable Energy Laboratory}\\ 
{Marc Day}, \textit{National Renewable Energy Laboratory}\\ 
\vspace{-0.1 in} \\ 
\noindent Internal combustion engines are ubiquitous in electrical power production and transportation systems. To best design these systems for energy efficiency and reduced emissions, a low Mach number turbulent reacting flows code, PeleLMeX, was developed for use in simulating the processes present in an internal combustion engine on an exascale machine. PeleLMeX features the capability of injecting liquid fuels into a gaseous ambient in the form of spray particles and handling the vaporization of said liquid fuel. In order to prepare this code for use on an exascale machine, validation tests demonstrating the ability of PeleLMeX to correctly simulate the coupling between liquid spray particles and the ambient gas were conducted. These tests demonstrate the capability of PeleLMeX to correctly simulate spray particles being advected by a uniform ambient flow, ejected from symmetric, counter-rotating, decaying Taylor-Green vortices, and evaporating and spreading when initialized in a realistic configuration.  \\ 
\begin{flushright}\vspace{-0.2 in}\hyperlink{toc}{Back to table of contents}\end{flushright}\vspace{-0.2 in}
\hypertarget{SamuelWhitman}{\subsection*{\color{CUGOLD} Pressure Gradient Tailoring Effects On Vorticity Dynamics In The Near-Wake Of Bluff-Body Stabilized Flames}} \vsp 
\underline{Samuel Whitman}, \textit{Mechanical Engineering, University Of Colorado Boulder}\\ 
\vspace{-0.1 in} \\ 
\noindent TBD - submitted by Tyler Souders \\ 
\begin{flushright}\vspace{-0.2 in}\hyperlink{toc}{Back to table of contents}\end{flushright}\vspace{-0.2 in}
\hypertarget{TylerSouders}{\subsection*{\color{CUGOLD} Pressure Gradient Tailoring Effects For Bluff-Body Stabilized Flames Subjected To Free-Stream Turbulence}} \vsp 
\underline{Tyler Souders}, \textit{Mechanical Engineering, University Of Colorado Boulder}\\ 
{Samuel Whitman}, \textit{Mechanical Engineering, University Of Colorado Boulder}\\ 
{Michael Meehan}, \textit{Mechanical Engineering, University Of Colorado Boulder}\\ 
{Peter Hamlington}, \textit{Mechanical Engineering, University Of Colorado Boulder}\\ 
\vspace{-0.1 in} \\ 
\noindent We showcase work in progress results for a bluff body stabilized propane flame subjected to strong inlet turbulence. Turbulence is generated using a synthetic method in order to offer realistic but reproducible flow profiles. We show early time-averaged results for the flame characteristics observed in a real-world channel. \\ 
\begin{flushright}\vspace{-0.2 in}\hyperlink{toc}{Back to table of contents}\end{flushright}\vspace{-0.2 in}
\hypertarget{LauraClark}{\subsection*{\color{CUGOLD} Dispersion Of Non-Spherical Particles By Waves And Currents}} \vsp 
\underline{Laura Clark}, \textit{Civil & Environmental Engineering, Stanford University (Planning To Move To Colorado; Melissa Moulton Suggested I Attend)}\\ 
{Michelle Dibenedetto}, \textit{Mechanical Engineering, University Of Washington}\\ 
{Nicholas Ouellette}, \textit{Civil And Environmental Engineering, Stanford University}\\ 
{Jeffrey Koseff}, \textit{Civil And Environmental Engineering, Stanford University}\\ 
\vspace{-0.1 in} \\ 
\noindent Motivated by the problem of microplastics in the ocean, we experimentally investigated the dispersion of non-spherical particles in a wave-current flow. We released millimeter-scale, negatively buoyant particles of different shapes into a flow both with and without waves and analyzed their landing positions to quantify how much they had dispersed while in the flow. We found that the presence of waves significantly increased the dispersion of the particles, and that the magnitude of this increase depended on particle shape and size. Specifically, the dispersion of rods was more increased by the waves than the dispersion of disks, and smaller particles had greater relative dispersion than larger particles. Although all of the particles traveled farther in the presence of waves, the increase in dispersion was much greater than could be explained solely by increased transport distance. \\ 
\begin{flushright}\vspace{-0.2 in}\hyperlink{toc}{Back to table of contents}\end{flushright}\vspace{-0.2 in}
\hypertarget{JaimeHerriott}{\subsection*{\color{CUGOLD} Small-Scale Variations In Ocean Acidity Using A Large Eddy Simulation}} \vsp 
\underline{Jaime Herriott}, \textit{Atmospheric And Oceanic Sciences, University Of Colorado Boulder}\\ 
\vspace{-0.1 in} \\ 
\noindent Abstract We investigate the small-scale effects of Langmuir turbulence on ocean acidity indicators pH and calcium carbonate saturation state () using the National Center for Atmospheric Research Large Eddy Simulation (NCAR LES) model with carbonate chemistry.  We analyze output from four simulations of the surface ocean boundary layer with varying intensities of Langmuir turbulence.  We find that Langmuir turbulence decreases the spatial variability of surface ocean acidity indicators pH and .  When no turbulence is present, pH and  have high spatial variability and elevated mean values.  As turbulence is intensified, the mean values for both variables decrease and variance decreases.  Increased Langmuir turbulence generates deeper mixing and a deepening of the carbocline (the depth of the maximum vertical carbon gradient).  Our results suggest that calcium carbonate-shelled organisms experience high variability in ocean acidity indicators pH and , which could affect both their shell-building (pH) and shell maintenance ().  As ocean acidification continues to plague marine organisms, these findings can help shape our understanding of organisms’ exposure to acidification variability during their life history.   \\ 
\begin{flushright}\vspace{-0.2 in}\hyperlink{toc}{Back to table of contents}\end{flushright}\vspace{-0.2 in}
\hypertarget{FedericoMunicchi}{\subsection*{\color{CUGOLD} Harnessing Buoyancy-Driven Instability To Enhance Thermal Membrane Desalination}} \vsp 
\underline{Federico Municchi}, \textit{Mechanical Engineering, Colorado School Of Mines}\\ 
{Nils Tilton}, \textit{Mechanical Engineering, Colorado School Of Mines}\\ 
\vspace{-0.1 in} \\ 
\noindent Membrane Distillation (MD) is an emerging method of desalinating complex wastewaters and hypersaline brines. MD faces two technological challenges.1) temperature polarization, is the cooling of the feed in a thermal boundary layer that forms at the membrane due heat lost to evaporation. This reduces the local rate of permeate production. 2) concentration polarization, is the accumulation of solutes in a simultaneous concentration boundary layer on the membrane. This leads to precipitation of solutes. \\ 
\noindent  \\ 
\noindent Surprisingly, no prior work considers that temperature and concentration polarization increase the feed density near the membrane. We show that with gravity properly oriented, this can trigger a buoyancy-driven instability in which plumes of cool, solute-rich, feed sink away from the membrane. This brings warm, low-concentration, feed to the membrane, mitigating temperature and concentration polarization.We perform computational fluid dynamics simulations to explore the dependence of buoyancy-driven instability on the operating and feed conditions and show how to sustain the instability over long membrane surfaces. \\ 
\noindent  \\ 
\noindent  \\ 
\noindent This work was supported by the Department of Energy (DOE) (Award No. DE-EE0008391) and the Sustainable LA Grand Challenge (EMVH). \\ 
\begin{flushright}\vspace{-0.2 in}\hyperlink{toc}{Back to table of contents}\end{flushright}\vspace{-0.2 in}
\hypertarget{MaryMcguinn}{\subsection*{\color{CUGOLD} Interactions Between Physical Processes And Carbonate Chemistry In The Oceanic Mixed Layer}} \vsp 
\underline{Mary Mcguinn}, \textit{Mechanical Engineering, University Of Colorado Boulder}\\ 
{Skyler Kern}, \textit{Mechanical Engineering, University Of Colorado Boulder}\\ 
{Katherine Smith}, \textit{Los Alamos National Laboratory}\\ 
{Nicole Lovenduski}, \textit{Atmospheric And Oceanic Sciences, University Of Colorado Boulder}\\ 
{Kyle Niemeyer}, \textit{Mechanical Engineering, Oregon State University}\\ 
{Peter Hamlington}, \textit{Mechanical Engineering, University Of Colorado Boulder}\\ 
\vspace{-0.1 in} \\ 
\noindent Ocean tracers play critical roles in the global carbon cycle and climate. These tracers evolve primarily in the oceanic mixed layer where gas exchange occurs and light is plentiful. There is substantial heterogeneity in tracer spatial distributions, and the effects of submesoscale turbulence remain incompletely understood. In this study, large eddy simulations are used to examine the effects of wind- and wave-driven turbulence, diurnal forcing, and wave breaking on carbonate chemistry in the oceanic mixed layer at submesoscales. Simulations are performed to determine the effects of varied physical processes on the air-sea carbon dioxide flux and amount of total dissolved inorganic carbon. The wave-averaged Boussinesq equations are solved in the simulations using pseudo-spectral and finite difference methods in the horizontal and vertical directions, respectively. The carbonate chemistry system consists of seven reacting species and is solved using a Runge-Kutta-Chebyshev integration scheme. Results are presented for the evolution and steady-state properties of each chemical species. Non-dimensional parameters are used to classify the results. Implications of these results for Earth system models are outlined, and an outlook for future research directions is provided. \\ 
\begin{flushright}\vspace{-0.2 in}\hyperlink{toc}{Back to table of contents}\end{flushright}\vspace{-0.2 in}
\hypertarget{SkylerKern}{\subsection*{\color{CUGOLD} Automatic Parameter Estimation Study For A Coupled Biophysical Ocean Model}} \vsp 
\underline{Skyler Kern}, \textit{Mechanical Engineering, University Of Colorado Boulder}\\ 
{Mary Mcguinn}, \textit{Mechanical Engineering, University Of Colorado Boulder}\\ 
{Peter Hamlington}, \textit{Mechanical Engineering, University Of Colorado Boulder}\\ 
{Kyle Niemeyer}, \textit{School Of Mechanical, Industrial, And Manufacturing Engineering, Oregon State University}\\ 
{Nicole Lovenduski}, \textit{Atmospheric And Oceanic Sciences, University Of Colorado Boulder}\\ 
{Nadia Pinardi}, \textit{Physics And Astronomy, University Of Bologna}\\ 
{Katherine Smith}, \textit{Los Alamos National Laboratory}\\ 
\vspace{-0.1 in} \\ 
\noindent Ocean biogeochemical (BGC) processes cannot be described using first principles forcing modelers to combine theory, empiricism, and numerical fitting. BGC systems are included in earth and regional scale ocean simulations by developing models that are able to capture crucial dynamics from observational data sets. A crucial part of producing representative models is correctly estimating the potentially large set of free parameters to better fit the data. This study addresses this challenge using random sampling and gradient-based optimization to estimate parameters for the 17 state variable implementation of the Biogeochemical Flux Mode (BFM17). For the purpose of this demonstration, two sites corresponding to the Bermuda Atlantic Time Series and the Hawaii Ocean Time Series have been simulated by coupling BFM17 to the one-dimensional implementation of the Princeton Ocean Model. The random sampling and optimization are performed with DAKOTA, an open-source numerical toolbox. The resulting multi-objective, multi-site parameter estimation is shown to give improved agreement between the simulated and observational data, suggesting the value of the demonstrated approach for future parameter estimation efforts with other BGC models.  \\ 
\begin{flushright}\vspace{-0.2 in}\hyperlink{toc}{Back to table of contents}\end{flushright}\vspace{-0.2 in}
\hypertarget{MalikJordan}{\subsection*{\color{CUGOLD} Tools For Analyzing And Reducing Ocean Biogeochemical And Transport Models}} \vsp 
\underline{Malik Jordan}, \textit{Mechanical Engineering, Oregon State University}\\ 
{Kyle Niemeyer}, \textit{Mechanical Engineering, Oregon State University}\\ 
\vspace{-0.1 in} \\ 
\noindent Biogeochemical (BGC) tracers in the upper ocean play an important role in the global carbon cycle and climate system. Submesoscale eddies and small-scale Langmuir turbulence in the upper ocean can occur at similar time scales to BGC processes, indicating the potential for coupling between the two phenomena. Additionally, turbulent processes can result in spatial and temporal heterogeneity in tracer distributions while also influencing near-surface flushing and mixed layer depth. The Biogeochemical Flux Model (BFM) simulates BGC and lower trophic level ecology of the marine environment. It consists of 56 state equations, resulting in high computational costs if coupled to a physical, multidimensional ocean model. Here we present an automated model reduction methodology which eliminates unnecessary components while retaining accuracy in target components of interest. Additionally, we developed a Python-based, 1D transport solver based on the Princeton Ocean Model, an ocean circulation model used to examine the time evolution of physical variables. We will couple these tools to automatically reduce and optimize BGC and transport models and compare against real-world measurements. Finally, we discuss future work that includes developing a generalizable BGC model framework. \\ 
\begin{flushright}\vspace{-0.2 in}\hyperlink{toc}{Back to table of contents}\end{flushright}\vspace{-0.2 in}
\hypertarget{EmanYahia}{\subsection*{\color{CUGOLD} Lattice Boltzmann Simulations Of Magnetohydrodynamic Flows Bounded By Electrically Conducting Walls}} \vsp 
\underline{Eman Yahia}, \textit{Mechanical Engineering, University Of Colorado Denver}\\ 
{Kannan Premnath}, \textit{Mechanical Engineering, University Of Colorado Denver}\\ 
\vspace{-0.1 in} \\ 
\noindent Magnetohydrodynamic (MHD) flow of liquid metals through conduits plays an important role in the proposed systems for harnessing fusion energy, and various other engineering and scientific problems. The interplay between the magnetic fields and the fluid motion gives rise to complex flow physics, which depends on the electrical conductivity of the bounding walls. An effective approach to represent the latter is via the Shercliff boundary condition for thin conducting walls relating the induced magnetic field and its wall-normal gradient at the boundary via a parameter referred to as the wall conductance ratio. Here, we present two new boundary schemes that enforce the Shercliff boundary condition in the LB schemes for MHD bounded by conducting walls. One approach is constructed using a link-based formulation involving a weighted combination of the bounce back and anti-bounce back of the distribution function for the magnetic field and the other approach involves an on-node moment-based implementation. Numerical validations of the boundary schemes for body force or shear driven MHD flows for a wide range of the values of the wall conductance ratio and their second order grid convergence are demonstrated. \\ 
\begin{flushright}\vspace{-0.2 in}\hyperlink{toc}{Back to table of contents}\end{flushright}\vspace{-0.2 in}
\hypertarget{NathanJarvey}{\subsection*{\color{CUGOLD} Application Of Boundary Layer Theory From Fluid Mechanics To Multicomponent Energy Storage Problems}} \vsp 
\underline{Nathan Jarvey}, \textit{Chemical And Biological Engineering, University Of Colorado Boulder}\\ 
{Filipe Henrique}, \textit{Chemical And Biological Engineering, University Of Colorado Boulder}\\ 
{Nathan Jarvey}, \textit{Chemical And Biological Engineering, University Of Colorado Boulder}\\ 
\vspace{-0.1 in} \\ 
\noindent Multicomponent electrolyte devices see vast use across the fields of energy storage and environmental remediation. Many of these devices use both reduction/oxidation reactions and electrical double layers. However, models which account for both effects often assume that the double layers and reactions are independent. As such, the balance of transport processes which governs the dynamics of coupled double layers and reactions remains unclear.  \\ 
\noindent  \\ 
\noindent In this talk, we present a physics-based theoretical framework to capture the dynamics of a model cell that couples the effects of double layers and electrochemical reactions. We use a perturbation analysis to solve the Poisson-Nernst-Planck equations, which enables us to predict spatial variations in ion concentration and potential at two distinct timescales: double layer charging and bulk diffusion. Our framework allows for any number of reactions and ions in the thin double layer limit. From our analysis, we discover that reaction rate directly affects double layer formation and the two processes cannot be treated independently. Furthermore, our model reveals that the thickness of electrical double layers is dependent on reaction rate, which implies that reactions can be used to control the capacitance of the overall system. \\ 
\noindent  \\ 
\begin{flushright}\vspace{-0.2 in}\hyperlink{toc}{Back to table of contents}\end{flushright}\vspace{-0.2 in}
\hypertarget{FilipeHenrique}{\subsection*{\color{CUGOLD} Applying The Principles Of Flow In Porous Media To Energy Storage Applications}} \vsp 
\underline{Filipe Henrique}, \textit{Chemical And Biological Engineering, University Of Colorado Boulder}\\ 
{Ankur Gupta}, \textit{Chemical And Biological Engineering, University Of Colorado Boulder}\\ 
{Pawel Zuk}, \textit{Physical Chemistry, Polish Academy Of Sciences, Warsaw, Pl}\\ 
\vspace{-0.1 in} \\ 
\noindent Electrolyte transport in charged confined geometries is critical to applications such as electrochemical carbon capture and energy storage. In these geometries, electric double layers form close to the boundaries and screen the electric field, influencing transport. Therein, the characteristic pore thickness is typically comparable to the double-layer thickness. Notwithstanding, the literature focuses on the cases of flat-plate electrodes and valence- and diffusivity-symmetric electrolytes. Accounting for arbitrary double-layer thickness and electrolyte asymmetry is important to optimizing electrochemical energy storage devices. \\ 
\noindent  \\ 
\noindent We proposed a novel model for ion transport in a pore for a binary electrolyte with arbitrary double-layer thickness, ionic valences, and ionic diffusivities, based on a linearization of the Poisson-Nernst-Planck equations. We show that diffusivity asymmetry drives nontrivial salt dynamics coupled to the charge dynamics even at low potentials, with potential applications to capacitive deionization. We quantify the control of valences and diffusivities over the two timescales governing the charging dynamics of asymmetric electrolytes at a constant capacitance, an absent feature in widely used transmission line representations. \\ 
\begin{flushright}\vspace{-0.2 in}\hyperlink{toc}{Back to table of contents}\end{flushright}\vspace{-0.2 in}
\hypertarget{AminTaziny}{\subsection*{\color{CUGOLD} A Multi-Scale Framework To Model The Fluid Dynamics Of Electrospray Thrusters}} \vsp 
\underline{Amin Taziny}, \textit{Aerospace Engineering, University Of Colorado Boulder}\\ 
{Wai Hong Ronald Chan}, \textit{Aerospace Engineering, University Of Colorado Boulder}\\ 
{Iain Boyd}, \textit{Aerospace Engineering, University Of Colorado Boulder}\\ 
\vspace{-0.1 in} \\ 
\noindent Electrospray thrusters are a promising means of micropropulsion for spacecraft due to their high thrust precision and range of specific impulses. A subset of electrosprays---ionic liquid ion sources (ILISs)---operate by exposing room temperature molten salts to strong potential gradients where extracted species produce thrust on the order of micronewtons. To model these systems, a multi-scale framework spanning continuum and kinetic regimes is presented. These regimes are represented by a coupled pair of computational domains: one at the emitter tip where ion evaporation occurs and the other at the plume where downstream ions are advected. These domains are denoted as the Extraction Region and Plume Region, respectively. In the Extraction Region, a multi-grid electrohydrodynamic model solves the Laplace and Stokes equations, where an iterative routine enforces surface charge evolution and stress balance at the meniscus. In the Plume Region, a combined particle-in-cell/direct simulation Monte Carlo (PIC/DSMC) approach treats far-field ion transport. The framework seeks to elucidate life-limiting mechanisms of electrospray thrusters by relating performance parameters across a wide range of spatial scales.     \\ 
\begin{flushright}\vspace{-0.2 in}\hyperlink{toc}{Back to table of contents}\end{flushright}\vspace{-0.2 in}
\hypertarget{ThomasKava}{\subsection*{\color{CUGOLD} Numerical Simulation Of Plasma Fueled Engines}} \vsp 
\underline{Thomas Kava}, \textit{Aerospace Engineering, University Of Colorado Boulder}\\ 
{Iain Boyd}, \textit{Aerospace Engineering, University Of Colorado Boulder}\\ 
{John Evans}, \textit{Aerospace Engineering, University Of Colorado Boulder}\\ 
\vspace{-0.1 in} \\ 
\noindent Numerical simulations are performed for a novel hypersonic air-breathing propulsion concept called the plasma fueled engine. Plasma fueled engines ionize the freestream air using high-energy electron beams. Then crossed electromagnetic fields generate a Lorentz force that accelerates the plasma through the engine. Thrust generation using electromagnetic body forces is advantageous to conventional chemical fuel-based propulsion systems because they are not limited to the stagnation pressures necessary for combustion. Thus, hypersonic propulsion is possible for high Mach number flight at high altitudes. Indeed, one-dimensional theoretical analyses of magnetohydrodynamic acceleration predict that the electromagnetic fields must be carefully designed for efficient acceleration at high Mach numbers. Both one- and three-dimensional simulations of preliminary plasma fueled engine geometries show that the electromagnetic fields must be tailored such that the Lorentz force overcomes the pressure gradient formed by Joule heating. This presentation will describe the numerical models used to analyze flows in plasma fueled engines. Furthermore, the fundamental physical phenomena necessary to understand the magnetohydrodynamic acceleration of hypersonic flows will be discussed.  \\ 
\begin{flushright}\vspace{-0.2 in}\hyperlink{toc}{Back to table of contents}\end{flushright}\vspace{-0.2 in}
\hypertarget{ReeceChurchill}{\subsection*{\color{CUGOLD} The Research And Motor Octane Numbers Of Liquified Petroleum Gas (Lpg) And Dimethyl Ether (Rdme) Blends}} \vsp 
\underline{Reece Churchill}, \textit{Mechanical Engineering, Colorado State University}\\ 
{Gokul Vishwanathan}, \textit{Propane Education And Research Council}\\ 
{Daniel Olsen}, \textit{Mechanical Engineering, Colorado State University}\\ 
{Bret Windom}, \textit{Mechanical Engineering, Colorado State University}\\ 
\vspace{-0.1 in} \\ 
\noindent               This presentation discusses an experimental study into the Motor and Research Octane Numbers of Liquified Petroleum Gas (LPG) blended with concentrations ranging from zero to thirty percent of renewable Dimethyl Ether (rDME). Data for the gaseous fuel blends of LPG and DME were obtained by utilizing the ASTM standards for RON and MON testing of liquid fuels. The Cooperative Fuels Research (CFR) engine’s intake system was modified to accept both gaseous fuels and liquid reference fuels through the three-bowl carburetor system. The LPG/DME blend composition was measured before and after engine testing through gas chromatography. As the concentration of DME increases, it results in a reduction of the octane number for the LPG and DME blend. \\ 
\noindent  \\ 
\noindent  \\ 
\begin{flushright}\vspace{-0.2 in}\hyperlink{toc}{Back to table of contents}\end{flushright}\vspace{-0.2 in}
\hypertarget{GrahamPash}{\subsection*{\color{CUGOLD} Towards Uncertainty Propagation For Data-Driven Turbulence Closure Models}} \vsp 
\underline{Graham Pash}, \textit{High Performance Algorithms And Complex Fluids, National Renewable Energy Laboratory}\\ 
{Malik Hassanaly}, \textit{High Performance Algorithms And Complex Fluids, National Renewable Energy Laboratory}\\ 
{Shashank Yellapantula}, \textit{High Performance Algorithms And Complex Fluids, National Renewable Energy Laboratory}\\ 
{Michael Mueller}, \textit{Mechanical And Aerospace Engineering, Princeton University}\\ 
\vspace{-0.1 in} \\ 
\noindent Large eddy simulation (LES) offers a path towards reducing the computational burden of direct numerical simulations (DNS) by resolving larger length scales and modeling the smallest scales. However, this filtering approach requires modeling closure terms to account for the effects at the sub-filter scale (SFS). Many closure model forms have been posited based on effects such as the ratio of time or length scales, however with the vast amounts of data available from DNS, there are unique opportunities to leverage data-driven modeling techniques. Albeit flexible, data-driven models still depend on the dataset and the functional form of the model chosen. Disseminating such models requires reliable uncertainty estimates in the data-informed and out-of-distribution regimes. To quantify uncertainties in both regimes, we employ Bayesian neural networks which are able to capture both epistemic and aleatoric uncertainties. Here, the focus is on modeling the progress variable scalar dissipation rate which plays a key role in the modeling of turbulent premixed flames. We will present various methods for incorporating prior knowledge into the resultant model. A pathway towards uncertainty propagation through the LES using this class of models will be outlined. \\ 
\begin{flushright}\vspace{-0.2 in}\hyperlink{toc}{Back to table of contents}\end{flushright}\vspace{-0.2 in}
\hypertarget{RiccardoBalin}{\subsection*{\color{CUGOLD} Online Learning Of Turbulence Closure Model At Scale}} \vsp 
\underline{Riccardo Balin}, \textit{Argonne National Laboratory}\\ 
{Filippo Simini}, \textit{Argonne National Laboratory}\\ 
{Aviral Prakash}, \textit{Aerospace Engineering, University Of Colorado Boulder}\\ 
{Basu Parmar}, \textit{Aerospace Engineering, University Of Colorado Boulder}\\ 
{John A. Evans}, \textit{Aerospace Engineering, University Of Colorado Boulder}\\ 
{Kenneth E. Jansen}, \textit{Aerospace Engineering, University Of Colorado Boulder}\\ 
\vspace{-0.1 in} \\ 
\noindent Data-driven approaches, such as neural networks (NN), for large eddy simulation (LES) and wall-modeled LES have been gaining popularity as they offer encouraging results for improved predictive capacity over traditional closure models. However, since NN for LES closures must be trained on instantaneous high-fidelity turbulent data, learning from high Reynolds number and complex flows requires multi-terabyte databases to store the training data. This limitation is resolved by performing online (in situ) learning, wherein the NN model is trained concurrently with the flow simulation producing the data and thus eliminating the need to store large training datasets on disk. This talk will cover the software infrastructure being developed to perform online learning at scale on current and future supercomputers, as well as its application to learning a NN model of the sub-grid stress tensor from a wide range of flows. \\ 
\begin{flushright}\vspace{-0.2 in}\hyperlink{toc}{Back to table of contents}\end{flushright}\vspace{-0.2 in}
\hypertarget{AlbertoOlmo}{\subsection*{\color{CUGOLD} Physics-Conforming Turbulent Flow Simulations Compression Approach}} \vsp 
\underline{Alberto Olmo}, \textit{National Renewable Energy Laboratory}\\ 
{Ahmed Zamzam}, \textit{National Renewable Energy Laboratory}\\ 
{Andrew Glaws}, \textit{National Renewable Energy Laboratory}\\ 
{Ryan King}, \textit{National Renewable Energy Laboratory}\\ 
\vspace{-0.1 in} \\ 
\noindent With the growing size of turbulent flow simulations, data compression approaches become an utmost importance to analyze, visualize, or restart the simulations. Recently, in-situ autoencoder-based compression approaches were proposed and shown to be effective in producing dimensionality-reduced representations of flow simulations. However, these approaches tend to focus solely on training the model based on sample quality losses while not taking advantage of the physical properties of turbulent flows. In this paper, we show that training autoencoders with additional physics-informed regularizations, e.g., incompressibility and preservation of enstrophy, improves a baseline model without such regularizations in three ways: i) upon inspection of the trained compression filters of the neural network, we identify changes in the convolutions due to the inclusion of the physics-informed terms ii) the compressions prove to be more physics-conforming to homogeneous isotropic turbulences of different Reynolds numbers given that these adhere to both the divergence free condition and preservation of enstrophy without trading off reconstruction quality, and iii) as a performance byproduct, training shows to converge 4 times faster than the baseline model. \\ 
\begin{flushright}\vspace{-0.2 in}\hyperlink{toc}{Back to table of contents}\end{flushright}\vspace{-0.2 in}
\hypertarget{TahaniAlsadik}{\subsection*{\color{CUGOLD} Multiphase Pseudopotential Lattice Boltzmann Model Using Multiple Relaxation Times For Phase Change Problem}} \vsp 
\underline{Tahani Alsadik}, \textit{Mechanical Engineering, University Of Colorado Denver}\\ 
{Samuel Welch}, \textit{Mechanical Engineering, University Of Colorado Denver}\\ 
\vspace{-0.1 in} \\ 
\noindent  \\ 
\noindent We numerically investigate and develop a pseudopotential lattice Boltzmann model (LB) with multiple relaxation times (MRT) combined with the energy equation to simulate thermal multiphase flows with phase change (e.g., nucleate boiling). The use of the pseudopotential LB model is an attractive approach to studying heat transfer phenomena for two phase flows due to its simplicity. Thermodynamic consistency is realized by forcing mechanical equilibrium to agree with the Maxwell construction. The energy equation is solved using regular second-order finite differences. This thesis proposal introduces a new formulation of the force term in a multiple relaxation time (MRT)-lattice Boltzmann (LB) model for the axisymmetric class. The proposed model is based on a simpler and more stable orthogonal moment basis, while the use of MRT introduces additional flexibility, and the model maintains the stream-collide procedure. The pseudopotential LB model is numerically validated via the simulations of static drop, as well as with the simulation of the Stephan problem for 1-D phase change and metastable boiling state. We find that a deceasing in the density of liquid below the equilibrium value can lead to the metastable state . \\ 
\begin{flushright}\vspace{-0.2 in}\hyperlink{toc}{Back to table of contents}\end{flushright}\vspace{-0.2 in}
\hypertarget{KiranEiden}{\subsection*{\color{CUGOLD} Using Machine-Learned Manifolds To Simplify The State Spaces Of Combustion Simulations}} \vsp 
\underline{Kiran Eiden}, \textit{National Renewable Energy Laboratory}\\ 
{Bruce Perry}, \textit{Fuels And Combustion Science, National Renewable Energy Laboratory}\\ 
{Marc Day}, \textit{Fuels And Combustion Science, National Renewable Energy Laboratory}\\ 
\vspace{-0.1 in} \\ 
\noindent Simulations of reacting flows often need to utilize complex reaction models in order to accurately predict the thermochemical evolution of the flow. These models are expensive to evaluate, and can severely limit the scale and scope of the simulations. It is unsurprising, then, that methods for reducing the computational cost of these models have been the subject of much prior work in reacting flows. One method that has been explored to some degree in the context of chemical combustion is manifold learning. Manifold learning is based on the principle that the set of quantities describing the full thermochemical state of a combustion system can be estimated using a much smaller set of parameters. The combustion system can be evolved on the low-dimensional manifold defined by these parameters, approximating the evolution of the more complex system at reduced computational cost. Here we present a novel approach to manifold learning that uses artificial neural networks to derive the mappings between the low and high-dimensional state space. We demonstrate the potential of this approach by applying it to turbulent combustion simulations that use PeleLMeX reacting flow solver. We also discuss factors affecting the performance of this modeling approach. \\ 
\begin{flushright}\vspace{-0.2 in}\hyperlink{toc}{Back to table of contents}\end{flushright}\vspace{-0.2 in}
\hypertarget{WilliamSchupbach}{\subsection*{\color{CUGOLD} Central Moment Lattice Boltzmann Methods For Multiphase Flows  Driven By Variable Surface Tension Effects Using High Performance Computing}} \vsp 
\underline{William Schupbach}, \textit{Mechanical Engineering, University Of Colorado Denver}\\ 
{Bashir Elbousefi}, \textit{Mechanical Engineering, University Of Colorado Denver}\\ 
{Kannan Premnath}, \textit{Mechanical Engineering, University Of Colorado Denver}\\ 
\vspace{-0.1 in} \\ 
\noindent Multiphase fluid flows occur frequently in nature, and in industrial applications. The presence of surfactants in these flows can significantly alter the interfacial dynamics and are often exploited to enhance wetting and de-wetting processes by either supporting or suppressing coalescence of bubbles and drops in the dispersed phase. On the other hand, self-re-wetting fluids exhibit nonlinear dependence of surface tension on temperature that cause fluid motions towards hotter regions, which has been exploited in thermal management applications. The common feature in these flows is a locally varying surface tension resulting in complex Marangoni convection. In this study we present a variety of these flows and discuss the effects of variable surface tension. Specifically, we show how surfactants effect the dynamics of bubbles as well as how self-rewetting fluids behave in a microchannel with sinusoidal heating profile. To simulate these flows, we use  robust central moment lattice Boltzmann schemes involving three distribution functions to solve the fluid motion and the energy transport, and track interfaces via a phase-field model. Furthermore, we utilize high performance computing with the MPI library for parallel simulations on the new Alderaan cluster at CU Denver. \\ 
\begin{flushright}\vspace{-0.2 in}\hyperlink{toc}{Back to table of contents}\end{flushright}\vspace{-0.2 in}
\hypertarget{ArkavaGanguly}{\subsection*{\color{CUGOLD} A Theoretical Framework To Understand Diffusiophoretic Self-Propulsion Of Slender Bent Rods}} \vsp 
\underline{Arkava Ganguly}, \textit{Chemical And Biological Engineering, University Of Colorado Boulder}\\ 
{Ankur Gupta}, \textit{Chemical And Biological Engineering, University Of Colorado Boulder}\\ 
\vspace{-0.1 in} \\ 
\noindent There have been several experimental attempts to control the motion of phoretic microswimmers by tuning both the surface chemistry and particle geometry. Over the last two decades, theoretical studies have focused mostly on understanding aspects of self-diffusiophoretic motion for spheroidal and cylindrical geometries. \\ 
\noindent  \\ 
\noindent In recent years, experiments have demonstrated rich particle-fluid dynamics when geometric asymmetry is introduced in diffusiophoretic systems. The standard boundary element method can describe the hydrodynamics for arbitrarily shaped particles but becomes computationally expensive for particles with a high aspect ratio. Current asymptotic theories for such slender geometries in phoretic flows are inadequate in understanding the motion of bent rods with no axial symmetry. \\ 
\noindent  \\ 
\noindent In our proposed framework, using boundary layer theory we explain the diffusiophoretic interactions at a continuum scale and then rely on reciprocal relations to obtain particle velocity. We see that the particles would turn continuously or translate with a stable orientation depending on the interplay of phoretic and hydrodynamic interactions. Our framework can predict a priori particle trajectories by understanding these interactions. \\ 
\begin{flushright}\vspace{-0.2 in}\hyperlink{toc}{Back to table of contents}\end{flushright}\vspace{-0.2 in}
\hypertarget{RituRaj}{\subsection*{\color{CUGOLD} Colloidal Banding Of Diffusiophoretic Particles In Two-Dimensional Solute Gradients}} \vsp 
\underline{Ritu Raj}, \textit{Chemical And Biological Engineering, University Of Colorado Boulder}\\ 
{C. Wyatt Shields Iv}, \textit{Chemical And Biological Engineering, University Of Colorado Boulder}\\ 
{Ankur Gupta}, \textit{Chemical And Biological Engineering, University Of Colorado Boulder}\\ 
\vspace{-0.1 in} \\ 
\noindent It is known that diffusiophoretic particles tend to enrich in regions due to an induced concentration gradient, a phenomenon known as colloidal banding. Most investigations have studied how particles band in response to one-dimensional concentration gradients. However, the application of this phenomenon to fields such as microfluidics requires understanding how particles will band under two-dimensional solute gradients. \\ 
\noindent  \\ 
\noindent In this work, we present on how a two-dimensional solute concentration field, generated by sources and sinks, can program the distribution of diffusiophoretic particles. First, we investigate how changing the dipolar separation distance between a source and sink can locally enrich or deplete particles. An interplay between characteristic flux decay and diffusion timescales controls enrichment in dipole systems and leads to a separation distance with maximal particle enrichment. We then study how geometric asymmetry in systems with four sources and four sinks can be used to systematically control particle banding structure and optimize particle enrichment. Overall, our findings show how the geometric arrangement of sources and sinks, as well as the diffusion and flux decay timescales, can be utilized to engineer systems with controlled particle banding structures. \\ 
\begin{flushright}\vspace{-0.2 in}\hyperlink{toc}{Back to table of contents}\end{flushright}\vspace{-0.2 in}
\hypertarget{AshishSrivastava}{\subsection*{\color{CUGOLD} Experimental And Computational Analyses Of Drop Motion In Straight Microchannels}} \vsp 
\underline{Ashish Srivastava}, \textit{Chemical And Biological Engineering, University Of Colorado Boulder}\\ 
{Aditya Vepa}, \textit{Chemical And Biological Engineering, University Of Colorado Boulder}\\ 
{Gesse Roure}, \textit{Chemical And Biological Engineering, University Of Colorado Boulder}\\ 
{Alexander Zinchenko}, \textit{Chemical And Biological Engineering, University Of Colorado Boulder}\\ 
{Robert Davis}, \textit{Chemical And Biological Engineering, University Of Colorado Boulder}\\ 
\vspace{-0.1 in} \\ 
\noindent Studying the motion and deformation of droplets in straight microchannels is central to the design of droplet-based microfluidic systems, whose applications range from lab-on-chip systems to high-throughput microreactors. Within this context, individual branches of microfluidic systems often assume the form of straight segments with rectangular cross-sections. Thus, it is important to investigate how physical and geometrical parameters such as the capillary number Ca, viscosity ratio λ, and drop size affect the motion of a droplet in straight, rectangular microfluidic channels. In this work, we study the motion of such droplets through numerical simulations and experiments. To this end, we use a boundary-integral algorithm to simulate a droplet moving through an infinitely-long, straight, rectangular channel analyzing how physical parameters affected drop velocity. Increasing the Ca results in a faster and more deformable droplet, whereas an increase in λ results in a slower droplet. We have also designed and built a flow cell in order to reproduce our numerical results. The experimental data are analyzed using computer-vision algorithms and our initial results show a good agreement with our simulations. \\ 
\begin{flushright}\vspace{-0.2 in}\hyperlink{toc}{Back to table of contents}\end{flushright}\vspace{-0.2 in}
\hypertarget{GesseRoure}{\subsection*{\color{CUGOLD} Numerical Investigation Of Deformable Droplets In Complex Microchannels}} \vsp 
\underline{Gesse Roure}, \textit{Chemical And Biological Engineering, University Of Colorado Boulder}\\ 
{Alexander Zinchenko}, \textit{Chemical And Biological Engineering, University Of Colorado Boulder}\\ 
{Robert Davis}, \textit{Chemical And Biological Engineering, University Of Colorado Boulder}\\ 
\vspace{-0.1 in} \\ 
\noindent The study of drop motion in microchannels has many applications, ranging from drug targeting to micro-chemical reactors. Fundamental understanding of the physics of drop motion in bounded domains is important for designing such systems. To this end, this work develops a boundary-integral algorithm to investigate drop motion in three-dimensional microchannels. The algorithm is coupled with a moving frame that follows the droplet throughout its motion, resulting in shorter computational times. Our meshing algorithm allows for simulation of complex channel geometries, including the main ones that are used in microfluidic devices. The presence of the front and back panels produces extra tail formations. The infinite-depth limit is analyzed by comparing the results to the ones obtained by infinite-depth simulations. Moreover, to probe the inertial effects at moderate Reynolds numbers, the full Navier-Stokes equations are solved for the undisturbed flow; the solution is then tabulated and used as a boundary condition for the moving frame. We find that, for moderate Reynolds numbers, inertial effects on the undisturbed flow are small even for a more irregular geometry, meaning that inertial contributions arise only from the transience of drop motion and are probably small. \\ 
\begin{flushright}\vspace{-0.2 in}\hyperlink{toc}{Back to table of contents}\end{flushright}\vspace{-0.2 in}
\hypertarget{MortezaGarousi}{\subsection*{\color{CUGOLD} Numerical Modeling Of Encapsulated Microbubbles Using The Lattice Boltzmann Method}} \vsp 
\underline{Morteza Garousi}, \textit{Mechanical And Aerospace Engineering, University Of Colorado Colorado Springs}\\ 
{Michael Calvisi}, \textit{Mechanical And Aerospace Engineering, University Of Colorado Colorado Springs}\\ 
\vspace{-0.1 in} \\ 
\noindent Encapsulated microbubbles (EMBs) are on the order of 1-10 microns in size and are used in biomedical applications, such as ultrasound imaging and targeted drug delivery. When an EMB is exposed to ultrasound, the high compressibility of its gas core causes both spherical (volumetric) and nonspherical (shape) oscillations. In this work, we use the Shan-Chen multiphase lattice Boltzmann method (SCMLBM) to numerically simulate such complex bubble dynamics. One shortcoming of the original SCMLBM is its inability to simulate multiphase flows with high density ratios; however, we overcome this limitation by using the Carnahan-Starling (C-S) equation of state (EOS). In this presentation, we share some preliminary results obtained from investigating bubble dynamics using SCMLBM combined with the C-S EOS. First, the effect of the reduced temperature on the achievable density ratio is discussed. In addition, the result of the Young-Laplace test is presented for a specific value of the reduced temperature. To incorporate acoustic forcing into the numerical model, we apply an oscillatory fluid density and velocity at the domain boundaries. Finally, the combination of immersed boundary and lattice Boltzmann methods to treat solid, deformable interfaces of EMBs is discussed. \\ 
\begin{flushright}\vspace{-0.2 in}\hyperlink{toc}{Back to table of contents}\end{flushright}\vspace{-0.2 in}
\hypertarget{DaYang}{\subsection*{\color{CUGOLD} Performance Characterization Of A Laminar Aircraft Gas-Inlet}} \vsp 
\underline{Da Yang}, \textit{Mechanical Engineering, Clarkson University}\\ 
{Rainer Volkamer}, \textit{Department Of Chemistry And Cires And Atoc, University Of Colorado Boulder}\\ 
{Suresh Dhaniyala}, \textit{Mechanical And Aeronautical Engineering, Clarkson University}\\ 
{Lee Mauldin}, \textit{Atmospheric And Oceanic Sciences (Atoc), University Of Colorado Boulder}\\ 
{Prakriti Sardana}, \textit{Chemistry, Cires And Mechanical Engineering, University Of Colorado Boulder}\\ 
\vspace{-0.1 in} \\ 
\noindent Aircraft-based measurements allow for large-scale characterization of gas-phase atmospheric composition but these measurements are complicated by the challenges of sampling from high-speed flow.  Under such sampling conditions, the gas often enters the sample tube turbulently, resulting in potential contamination from the condensed-phase component and transport losses.  While a significant amount of research has gone into understanding aerosol sampling efficiency for aircraft inlets, a similar research investment has not been made for gas sampling.   Here, we analyze the performance of a laminar gas inlet to establish its performance as a function of operating conditions, including ambient pressure, freestream velocities, and sampling conditions.  Using computational fluid dynamics (CFD) modeling we simulate flow inside and outside the inlet to determine the extent of freestream turbulent interaction with the sample flow and its implication for gas sample transport.  The CFD results of flow features in the inlet are compared against wind-tunnel experiment results. With minor changes to the inlet design, we demonstrate that efficient transport of gas samples is possible with this inlet operating under high-speed flight conditions. \\ 
\begin{flushright}\vspace{-0.2 in}\hyperlink{toc}{Back to table of contents}\end{flushright}\vspace{-0.2 in}
\hypertarget{SamanthaSheppard}{\subsection*{\color{CUGOLD} Experimental Exploration Of 3D Attached Eddy Structures In The Surface Layer.}} \vsp 
\underline{Samantha Sheppard}, \textit{Aerospace Engineering, University Of Colorado Boulder}\\ 
{James Brasseur}, \textit{Aerospace Engineering, University Of Colorado Boulder}\\ 
{John Farnsworth}, \textit{Aerospace Engineering, University Of Colorado Boulder}\\ 
{Christos Vassilicos}, \textit{Laboratoire De Mecanique Des Fluides De Lille}\\ 
{Pierre Braganca}, \textit{Laboratoire De Mecanique Des Fluides De Lille}\\ 
{Christophe Cuvier}, \textit{Laboratoire De Mecanique Des Fluides De Lille}\\ 
\vspace{-0.1 in} \\ 
\noindent In order to better understand how the interaction of turbulent eddies with an impermeable surface drives the statistical scaling of eddies within the surface layer new high-dimensional experimental data resources are necessary. To this end, 3-dimensional energy-containing turbulent structures were captured and reconstructed within the surface layer of a turbulent boundary layer and track their local evolution. Cross-tunnel planes of time-resolved stereoscopic particle image velocimetry measurements were collected in a turbulent boundary layer with a thickness δ=28 cm and Re_θ=7680 in the LMFL wind tunnel. The resulting flow field was transformed into a quasi-3D spatial domain through locally applying Taylor’s frozen flow hypothesis to the time-resolved data. The error and practical limitations of this temporal to spatial transformation are assessed for this region of the flow utilizing a corresponding time-resolved streamwise plane. From this plane the error between the transformed streamwise dimension and the actual streamwise dimension is quantified. These results will provide a foundation for future analysis and comparison with experimental grid turbulence data collected at CU Boulder. \\ 
\begin{flushright}\vspace{-0.2 in}\hyperlink{toc}{Back to table of contents}\end{flushright}\vspace{-0.2 in}
\hypertarget{NilsWunsch}{\subsection*{\color{CUGOLD} Simulation Of Turbulent Incompressible Flows Using Immersogeometric Analysis}} \vsp 
\underline{Nils Wunsch}, \textit{Aerospace Engineering, University Of Colorado Boulder}\\ 
{Lise Noel}, \textit{Department Of Precision And Microsystems Engineering, Delft University Of Technology}\\ 
{Kurt Maute}, \textit{Aerospace Engineering, University Of Colorado Boulder}\\ 
\vspace{-0.1 in} \\ 
\noindent The finite element (FE) method is a commonly used approach to numerically solve incompressible flow problems. However, given its dependency on mesh quality, generating adequate body-fitted meshes is time consuming and often requires manual intervention. Immersed FE methods provide an avenue for avoiding expensive mesh generation. This presentation introduces a framework based on an immersed isogeometric FE method and its application to turbulent incompressible flows modeled with the Spalart-Allmaras model. \\ 
\noindent   \\ 
\noindent In this framework, the geometry of the flow domain, defined by level-set functions, is immersed into a structured tensor-product background mesh. The FE approximation is enriched to consistently interpolate the flow variables and to avoid spurious coupling of geometrically separate flow regions. B-spline basis functions are employed for improved continuity of the solution, and to enable local hierarchical refinement. Essential boundary conditions are imposed weakly via Nitsche’s method. The immersed method is stabilized using Ghost stabilization. The flow problems are stabilized using the SUPG and PSPG methods. \\ 
\noindent  \\ 
\noindent Solutions for benchmark problems will be presented alongside results for a geometrically complex, topology optimized heat exchanger involving conjugate heat transfer. \\ 
\begin{flushright}\vspace{-0.2 in}\hyperlink{toc}{Back to table of contents}\end{flushright}\vspace{-0.2 in}
\hypertarget{DiederikBeckers}{\subsection*{\color{CUGOLD} Discretization Error Analysis Of Convective Schemes For Large Eddy Simulations With Adaptive Mesh Refinement}} \vsp 
\underline{Diederik Beckers}, \textit{National Renewable Energy Laboratory}\\ 
{Marc Henry De Frahan}, \textit{National Renewable Energy Laboratory}\\ 
{Lucas Esclapez}, \textit{National Renewable Energy Laboratory}\\ 
{Michael Brazell}, \textit{National Renewable Energy Laboratory}\\ 
{Bruce Perry}, \textit{National Renewable Energy Laboratory}\\ 
{Michael Mueller}, \textit{Princeton University}\\ 
{Marc Day}, \textit{National Renewable Energy Laboratory}\\ 
\vspace{-0.1 in} \\ 
\noindent Computationally tractable simulations of highly turbulent flows in science and engineering applications such as combustion and wind energy often rely on large eddy simulations (LES). The range of resolved scales in these simulations is increased by computing techniques such as efficient, high-order numerical schemes and adaptive mesh refinement (AMR). Recent investigations found that estimates of grid quantities can be significantly affected when they are convected across the interfaces between coarse and fine grids, such as those present in AMR. This work presents an analysis of the discretization error of the convective schemes used in NREL's AMR-Wind and Pele simulation codes. This effort seeks to characterize these solvers to understand the discretization errors arising from coarse-fine and fine-coarse grid interfaces in AMR, and provide solution pathways to avoid or reduce these errors. Specifically, we investigate the discretization error for a one-dimensional passive scalar advection case and a two-dimensional convecting Taylor vortex. \\ 
\begin{flushright}\vspace{-0.2 in}\hyperlink{toc}{Back to table of contents}\end{flushright}\vspace{-0.2 in}
\hypertarget{ThomasCalascione}{\subsection*{\color{CUGOLD} Swirl Generation In Turbulent Jets: A Literature Review}} \vsp 
\underline{Thomas Calascione}, \textit{Aerospace Engineering, University Of Colorado Boulder}\\ 
{John Farnsworth}, \textit{Aerospace Engineering, University Of Colorado Boulder}\\ 
\vspace{-0.1 in} \\ 
\noindent Swirl generation is a class of techniques for producing rotation in an axial flow. The resulting jet gains enhanced mixing due to added fluid entrainment, which is advantageous in applications such as combustion where mixing is a direct driver of efficiency. Various methods of swirl generation have been previously studied and can be categorized as either mechanical or fluidic. The former includes methods such as turning vanes or rotating pipes with honeycomb structures to induce swirl through mechanically directing momentum into rotation. The latter consists of less invasive methods such as tangential jets which induce swirl via added mass flow at an angle relative to the streamwise direction. The fundamental characterization of the physics for these swirl generation methods has not been explored and compared thoroughly. Thus, future particle image velocimetry measurements will be carried out on select mechanically and fluidically induced swirling jet geometries to further explore the flow physics. The results from these experiments will be used to inform the development of a conceptualized precessing jet which is hypothesized to have an overall larger level of entrainment due to its enhanced unsteadiness compared with classical swirling jets. \\ 
\begin{flushright}\vspace{-0.2 in}\hyperlink{toc}{Back to table of contents}\end{flushright}\vspace{-0.2 in}
